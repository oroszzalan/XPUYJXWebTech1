\documentclass[a4paper,12pt]{article}

% XeLaTeX / LuaLaTeX alapú Unicode kezelés
\usepackage{fontspec} % Unicode fontok
\setmainfont{Lato}    % Google fontod, vagy bármilyen TTF/OTF
\usepackage{xcolor}
\usepackage{listings}
\usepackage{hyperref}
\usepackage{graphicx}
\usepackage{setspace}
\usepackage{ragged2e}
\usepackage{geometry}
\geometry{margin=2.5cm}
\setstretch{1.3}
\justifying

% Kód formázás
\lstset{
    basicstyle=\ttfamily\small,
    keywordstyle=\color{blue},
    commentstyle=\color{green!50!black},
    stringstyle=\color{red},
    breaklines=true,
    frame=single,
    numbers=left,
    numberstyle=\tiny\color{gray},
    showstringspaces=false
}
\author{}
\date{\today}

\begin{document}

\maketitle
\tableofcontents
\newpage

\section{Bevezetés}
Ez a dokumentáció a Lamborghini Revuelto megrendelő weboldal teljes kódját és funkcionalitását mutatja be. Az oldal célja, hogy a felhasználó kiválaszthassa a modellt, színt, extrákat, fizetési módot, dátumot, és megjelenítse a rendelés összefoglalóját.

\section{CSS Stílusok}
A CSS felelős az oldal kinézetéért, animációkért és interaktív elemekért.

\begin{lstlisting}[language=CSS]
@import url("https://fonts.googleapis.com/css2?family=Lato:wght@300;400;700;900&display=swap");

* { margin: 0; padding: 0; box-sizing: border-box; }
:root { --primary: #b07c0c; }

body {
    position: relative;
    height: 100vh;
    width: 100vw;
    font-family: 'Lato', sans-serif;
    font-size: 14px;
    color: #eee;
    background: black;
    overflow: hidden;
}

/* Fejléc és navigáció */
.header { display:flex; justify-content:space-between; padding:20px 40px; position:absolute; width:100%; top:0; z-index:1000; }
nav ul { display:flex; gap:30px; }
nav ul li a { color:#eee; font-size:20px; text-transform:capitalize; }
a:hover { color:var(--primary); }

/* Tartalom és animációk */
.content { display:flex; gap:20px; position:absolute; left:30px; top:16vh; transition:all 0.6s; }
.title h1 { font-size:50px; font-weight:800; text-transform:uppercase; animation:slideRight 0.7s cubic-bezier(0.52, 0.19, 0.3, 1.46); }
#subtitle p { margin-top:20px; font-size:25px; animation:slideRight 0.7s; }
button { font-family:'Lato'; background:transparent; border:none; color:#eee; cursor:pointer; transition:all 0.7s; letter-spacing:2px; animation:slideUp 0.6s; }
button:hover { transform:scale(1.1); }

.banner { height:calc(100vh - 100px); width:100%; position:absolute; top:0; left:0; }
.swiper-slide img { object-fit:cover; transform:scaleX(-1); }
.thumbsSwiper .swiper-slide:hover { transform:scale(1.1); }
.slide-title { font-size:90px; color:#eeeeee61; text-transform:uppercase; display:none; }
.buy-btn { font-size:30px; position:absolute; bottom:70px; left:49.3%; cursor:pointer; display:none; transition:all 0.5s ease-in-out; color:#000; z-index:3333; }

.overlay { display:none; height:100vh; width:100vw; position:fixed; padding:20px; background-color:#000000f7; z-index:1000000; animation:grow 0.7s; }

/* Animációk */
@keyframes slideLeft { 0% {opacity:0; transform:translateX(300%);} 100% {opacity:1; transform:translateX(0);} }
@keyframes slideRight { 0% {opacity:0; transform:translateX(-130px);} 100% {opacity:1; transform:translateX(0);} }
@keyframes slideUp { 0% {opacity:0; transform:translateY(200%);} 100% {opacity:1; transform:translateY(0);} }
@keyframes zoomIn { 0% {opacity:0; transform:scale(0);} 100% {opacity:1; transform:scale(1);} }
\end{lstlisting}

\section{HTML Szerkezet}
A weboldal HTML struktúrája tartalmazza a formot, videó háttér elemet, és a rendelés összefoglalót.

\begin{lstlisting}[language=HTML]
<!DOCTYPE html>
<html lang="hu">
<head>
<meta charset="UTF-8">
<title>Lamborghini Revuelto Megrendelés</title>
<script src="https://code.jquery.com/jquery-3.6.0.min.js"></script>
<link rel="stylesheet" href="styles/order.css">
</head>
<body>

<video autoplay muted loop id="background-video">
    <source src="kepek/revuelto.mp4" type="video/mp4">
</video>

<h1>Lamborghini Revuelto Megrendelés</h1>

<form id="carOrderForm">
    <label for="name">Név:</label>
    <input type="text" id="name" name="name">
    <span class="error-message" id="name-error"></span>

    <label for="email">Email:</label>
    <input type="email" id="email" name="email">
    <span class="error-message" id="email-error"></span>

    <label for="address">Cím:</label>
    <textarea id="address" name="address"></textarea>
    <span class="error-message" id="address-error"></span>

    <label for="model">Modellválasztás:</label>
    <div id="model-wrapper">
        <select id="model" name="model">
            <option value="">Válassz modellt...</option>
        </select>
    </div>
    <span class="error-message" id="model-error"></span>

    <label>Színek:</label>
    <input type="color" id="color" name="color">

    <label>Dátum kiválasztása:</label>
    <input type="date" id="date" name="date">

    <label>Extrák:</label>
    <label class="extra-option">
        <input type="checkbox" id="spoiler" name="extras" value="Spoiler"> Spoiler
    </label>
    <label class="extra-option">
        <input type="checkbox" id="sound" name="extras" value="High-End Hangrendszer"> High-End Hangrendszer
    </label>

    <label>Fizetési mód:</label>
    <span class="payment-option">
        <input type="radio" id="card" name="payment" value="Kártya">
        <label for="card">Kártya</label>
    </span>
    <span class="payment-option">
        <input type="radio" id="cash" name="payment" value="Készpénz">
        <label for="cash">Készpénz</label>
    </span>
    <span class="error-message" id="payment-error"></span>

    <button type="submit">Megrendelem</button>
</form>

<div id="order-summary"></div>

</body>
</html>
\end{lstlisting}

\section{JavaScript / jQuery Funkciók}
A JavaScript felelős a JSON adatbetöltésért, form validációért és a rendelés összefoglaló megjelenítéséért.

\begin{lstlisting}[language=JavaScript]
$(document).ready(function() {

    let carData = [];

    // JSON betöltése
    $.ajax({
        url: "ajax/Revuelto.json",
        dataType: "json",
        success: function(data) {
            carData = data.cars;
            data.cars.forEach(function(car) {
                $("#model").append(`<option value="${car.name}">${car.name}</option>`);
            });
        }
    });

    // Form validáció
    $("#carOrderForm").submit(function(e) {
        e.preventDefault();
        let valid = true;

        if($("#name").val().trim() === "") {
            $("#name").addClass("error");
            $("#name-error").text("Kérlek add meg a neved!");
            valid = false;
        } else {
            $("#name").removeClass("error");
            $("#name-error").text("");
        }

        const emailPattern = /^[^\s@]+@[^\s@]+\.[^\s@]+$/;
        if(!emailPattern.test($("#email").val())) {
            $("#email").addClass("error");
            $("#email-error").text("Érvényes email cím szükséges!");
            valid = false;
        } else {
            $("#email").removeClass("error");
            $("#email-error").text("");
        }

        if($("#address").val().trim() === "") {
            $("#address").addClass("error");
            $("#address-error").text("Add meg a címed!");
            valid = false;
        } else {
            $("#address").removeClass("error");
            $("#address-error").text("");
        }

        if($("#model").val().trim() === "") {
            $("#model").addClass("error");
            $("#model-error").text("Válassz modellt!");
            valid = false;
        } else {
            $("#model").removeClass("error");
            $("#model-error").text("");
        }

        if(!$("input[name='payment']:checked").val()) {
            $("#payment-error").text("Válassz fizetési módot!");
            valid = false;
        } else {
            $("#payment-error").text("");
        }

        if(valid) {
            const selectedModel = $("#model").val();
            const selectedCar = carData.find(car => car.name === selectedModel);
            const price = selectedCar ? selectedCar.price : "Ismeretlen";

            const extras = $("input[name='extras']:checked")
                                .map(function(){ return this.value; })
                                .get().join(", ") || "Nincs";

            const summary = `<h3>Rendelés összefoglaló:</h3>
                             <p>Név: ${$("#name").val()}</p>
                             <p>Email: ${$("#email").val()}</p>
                             <p>Cím: ${$("#address").val()}</p>
                             <p>Modell: ${selectedModel}</p>
                             <p><strong>Ár: ${price} Ft</strong></p>
                             <p>Szín: <span style="color:${$("#color").val()}">${$("#color").val()}</span></p>
                             <p>Dátum: ${$("#date").val()}</p>
                             <p>Extrák: ${extras}</p>
                             <p>Fizetés: ${$("input[name='payment']:checked").val()}</p>`;

            $("#order-summary").html(summary).fadeIn();
        }
    });
});
\end{lstlisting}

\section{Funkcionális Leírás}
\begin{itemize}
    \item \textbf{Videó háttér:} Automatikusan indul, némított videó.
    \item \textbf{Form validáció:} Ellenőrzi a kötelező mezőket és a helyes e-mail formátumot.
    \item \textbf{Modellek betöltése:} JSON fájlból történik, dinamikusan feltölti a legördülő listát.
    \item \textbf{Rendelés összefoglaló:} Dinamikusan megjeleníti a felhasználó választásait.
    \item \textbf{Animációk:} Gombok, elemek és overlay animációk CSS `@keyframes` segítségével.
\end{itemize}

\end{document}
