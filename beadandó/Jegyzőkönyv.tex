\documentclass[a4paper,12pt]{article}
\usepackage[magyar]{babel}
\usepackage[utf8]{inputenc}
\usepackage[T1]{fontenc}
\usepackage{geometry}
\geometry{left=25mm,right=25mm,top=25mm,bottom=25mm}
\usepackage{graphicx}
\usepackage{listings}
\usepackage{xcolor}
\usepackage{fancyhdr}
\usepackage{sectsty}
\usepackage{float}
\usepackage{caption}
\usepackage[utf8]{inputenc}

\pagestyle{fancy}
\fancyhf{}
\rhead{\thepage}
\cfoot{}

\sectionfont{\Large\bfseries}
\subsectionfont{\large\bfseries}
\subsubsectionfont{\bfseries}

\lstset{
  basicstyle=\ttfamily\small,
  keywordstyle=\color{blue},
  stringstyle=\color{red},
  commentstyle=\color{gray},
  numbers=left,
  numberstyle=\tiny,
  stepnumber=1,
  numbersep=10pt,
  backgroundcolor=\color{white},
  showspaces=false,
  showstringspaces=false,
  frame=single,
  tabsize=2,
  captionpos=b,
  breaklines=true,
  language=HTML,
  morekeywords={class, id, onclick}
}

\title{\Huge\textbf{JEGYZŐKÖNYV}}
\author{}
\date{}

\begin{document}

\begin{titlepage}
    \centering
    \vspace*{3cm}
    {\Huge\bfseries JEGYZŐKÖNYV\par}
    \vspace{2cm}
    {\LARGE Web Technológiák 1.\par}
    \vspace{1cm}
    {\LARGE Féléves feladat\par}
    \vspace{1cm}
    {\LARGE Sportautók – interaktív prezentáció\par}
    \vspace{3cm}
    {\large Készítette: \textbf{Orosz Zalán Zétény}\par}
    {\large Neptun kód: \textbf{XPUYJX}\par}
    {\large Dátum: 2025. december\par}
    \vspace{2cm}
    {\large Miskolc, 2025\par}
\end{titlepage}

\tableofcontents
\newpage

\section{Bevezetés}
A modern weboldalak napjainkban már messze túlmutatnak a statikus tartalom megjelenítésén. A HTML5, CSS3, JavaScript és külső könyvtárak (Swiper.js, Font Awesome) kombinációjával látványos, interaktív, mobilbarát prezentációk készíthetők. A jelen feladat egy luxus sportautókat bemutató egyoldalas webalkalmazás, amely Swiper.js alapú carousel-t, animációkat, overlay videó lejátszót, dinamikus címváltoztatást és részletes műszaki adatlapot tartalmaz.

\section{Feladat leírása}
A dokumentum egy Lamborghini, McLaren és BMW sportautókat bemutató, látványos egyoldalas weboldalt ír le, amely HTML, CSS, JavaScript és a Swiper.js könyvtár segítségével készült. A cél egy prémium autóipari stílusú, teljesen reszponzív, sötét témájú prezentáció létrehozása, ahol a felhasználó böngészheti az autókat, megnézheti a videókat, és konfigurálásra továbbmehet.

\section{Weblap felépítése}
\subsection{Fejléc és főcím}
A fejléc fixen a tetején helyezkedik el, tartalmazza a hamburger ikont és nyelvválasztót. A főcím dinamikusan változik az aktív slide alapján (JavaScript \texttt{changeTitle()} függvény).

\subsection{Swiper carousel}
A Swiper.js két példánya fut: egy fő carousel (\texttt{.bannerSwiper}) és egy thumbnail sáv (\texttt{.thumbsSwiper}), amelyek szinkronban vannak.

\subsection{Interaktív elemek}
\begin{itemize}
    \item \texttt{EXPLORE} gomb – részletes nézet megnyitása
    \item \texttt{Külső} gomb – műszaki adatok megjelenítése
    \item \texttt{Buy} gomb – külső konfigurátor oldalra navigálás
    \item Lejátszás gomb – videó overlay megjelenítése
\end{itemize}

\newpage

\section{Scriptek}

\subsection{Autók adatai objektumban}
\begin{lstlisting}
var nevek = {
    0: { cim: "KONFIGURALD <br>REVUELTO", alcim: "FROM NOW ON", ... },
    1: { cim: "KONFIGURALD <br>HURACAN STO", ... },
    ...
};
\end{lstlisting}
Az objektum tárolja az egyes autók címét, alcímét, videó URL-jét és logóját.

\subsection{Swiper inicializálása}
\begin{lstlisting}
var thumbsSwiper = new Swiper(".thumbsSwiper", { ... });
const swiper = new Swiper(".bannerSwiper", {
    thumbs: { swiper: thumbsSwiper },
    speed: 1000,
    ...
});
\end{lstlisting}
A thumbnail és a fő carousel összekapcsolása a \texttt{thumbs} opcióval.

\subsection{Dinamikus cím és logó váltás}
\begin{lstlisting}
changeTitle = (index) => {
    const data = nevek[index];
    document.querySelector("#title").innerHTML = `<h1>${data.cim}</h1>`;
    document.querySelector("#subtitle").innerHTML = `<p>${data.alcim}</p>`;
    document.querySelectorAll('img.logo').forEach(img => 
        img.src = `kepek/${data.logo}`);
};

\end{lstlisting}

Listing 1. Dinamikus cím és logó váltás

\subsection{EXPLORE gomb működése}
Az \texttt{EXPLORE} gomb elrejti a thumbnail sávot és a főoldali címeket, hogy helyet adjon a részletes slide-nak.

\begin{lstlisting}
explore.forEach((el) => { 
    el.addEventListener("click", function() {
        thumbs.classList.add("hide");
        contents.forEach((e) => e.classList.add("hide"));
        headers.forEach((e) => e.classList.add("hide"));
        el.classList.add("hide");
    });
});
\end{lstlisting}

\subsection{Műszaki adatok megjelenítése}
A \texttt{Külső} gomb a \texttt{details} osztály hozzáadásával nyitja meg az adatlapot, és megjeleníti a \texttt{Buy} gombot.

\begin{lstlisting}
slideBtns.forEach(el => { 
    el.addEventListener("click", function() {
        document.querySelector(".swiper-slide-active")
            .classList.add("details");
        let buyBtn = activeSlide.querySelector(".buy-btn");
        if (buyBtn) buyBtn.style.display = "block";
    });
});
\end{lstlisting}

\subsection{Videó overlay kezelése}
\begin{lstlisting}
var showTrailer =() => {
    var index = swiper.activeIndex;
    videoContainer.innerHTML = `
        <video controls autoplay>
            <source src="kepek/${nevek[index].videoURL}" type="video/mp4">
        </video>`;
    overlay.classList.add("show");
};

var closeOverlay =() => {
    document.getElementById("video").pause();
    overlay.classList.remove("show");
};
\end{lstlisting}

Listing 2. Videó overlay megnyitása és bezárása

\subsection{Buy gomb – külső oldalra navigálás}
\begin{lstlisting}
var externalPages = ["revuelto.html", "sto.html", ...];

buyBtns.forEach((btn, index) => {
    btn.addEventListener("click", function() {
        window.location.href = externalPages[index];
    });
});
\end{lstlisting}

Listing 3. Külső konfigurátor oldalakra ugrás

\newpage

\section{A konfigurációs oldal (revuelto.html)}

\subsection{Háttérvideó}
\begin{lstlisting}
<video autoplay muted loop id="background-video">
    <source src="kepek/revuelto.mp4" type="video/mp4">
</video>
\end{lstlisting}
Autoplay, némított, végtelenített háttérvideó adja a prémium hatást.

\subsection{A rendelési űrlap felépítése}
Az űrlap tartalmazza:
\begin{itemize}
    \item Személyes adatok (név, email, cím)
    \item Modellválasztás (JSON-ból dinamikusan töltve)
    \item Színválasztó (\texttt{<input type="color">})
    \item Dátumválasztó
    \item Extrák (checkbox)
    \item Fizetési mód (radio)
\end{itemize}

\subsection{JSON-ból modellbetöltés}
\begin{lstlisting}
$.ajax({
    url: "ajax/Revuelto.json",
    dataType: "json",
    success: function(data) {
        data.cars.forEach(function(car) {
            $("#model").append(
                `<option value="${car.name}">${car.name}</option>`
            );
        });
    }
});
\end{lstlisting}

Listing 5. Modellválaszték betöltése JSON-ból

\newpage

\subsection{jQuery alapú validáció és összefoglaló}
A beküldéskor client-oldali validáció fut:

\begin{lstlisting}
$("#carOrderForm").submit(function(e) {
    e.preventDefault();
    let valid = true;

    
    
    if(valid) {
        const selectedCar = carData.find(car => car.name === $("#model").val());
        const price = selectedCar ? selectedCar.price : "Ismeretlen";

        const extras = $("input[name='extras']:checked")
            .map(function(){ return this.value; }).get().join(", ") || "Nincs";

        const summary = `<h3>Rendeles osszefoglalo:</h3>
                         <p><strong>Ar: ${price} Ft</strong></p>
                         <p>Szin: <span style="color:${$("#color").val()}">
                            ${$("#color").val()}</span></p>
                         <p>Extrak: ${extras}</p>`;
                         
        $("#order-summary").html(summary).fadeIn();
    }
});
\end{lstlisting}

Listing 6. Űrlap validáció és rendelés-összefoglaló

\subsection{Hibajelzés és vizuális visszajelzés}
Hibás mezőknél piros keret és egyedi hibaüzenet jelenik meg:
\begin{lstlisting}
.error { border: 2px solid red !important; }
.error-message { color: red; font-size: 0.9em; }
\end{lstlisting}

\newpage
\section{A Revuelto.json fájl példája}
\begin{lstlisting}[caption=Példa JSON struktúra]
{
    "cars": [
        {
            "name": "Revuelto Standard",
            "price": "189000000"
        },
        {
            "name": "Revuelto S",
            "price": "215000000"
        },
        {
            "name": "Revuelto Performante",
            "price": "248000000"
        }
    ]
}
\end{lstlisting}

Listing 7. Revuelto.json fájl

\section{Külső könyvtárak}
\begin{itemize}
    \item \textbf{Swiper.js v10} – carousel és thumbnail kezelés
    \item \textbf{Font Awesome 7} – ikonok
    \item \textbf{Google Fonts – Lato} – betűtípus
\end{itemize}

\section{Összegzés}
A weboldal teljes mértékben kihasználja a modern webtechnológiák nyújtotta lehetőségeket:
\begin{itemize}
    \item Dinamikus tartalom JavaScript objektumból
    \item Szinkronizált carousel és thumbnail
    \item Animált átmenetek CSS-ben és JavaScriptben
    \item Videó overlay autoplay-jal
    \item Külső konfigurátor oldalakra mutató linkek
    \item Teljesen reszponzív, prémium megjelenés
\end{itemize}

A kód tiszta, jól strukturált, könnyen bővíthető további autómodellekkel.

\end{document}