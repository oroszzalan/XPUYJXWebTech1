\documentclass[12pt,a4paper]{article}
% Betűtípus: Times New Roman
\usepackage{newtxtext,newtxmath}

% Magyar nyelv
\usepackage[utf8]{inputenc}
\usepackage[T1]{fontenc}
\usepackage[hungarian]{babel}

% Margók
\usepackage{geometry}
\geometry{margin=2.5cm}

% Másfeles sortáv
\usepackage{setspace}
\onehalfspacing

% Hiperhivatkozások és tartalomjegyzék
\usepackage{hyperref}
\hypersetup{
    colorlinks=true,
    linkcolor=blue
}

% Sorkizárás\ n\renewcommand\baselinestretch{1.5}

% Képkezelés
\usepackage{graphicx}
\usepackage{caption}

% Kódokhoz más betűtípus
\usepackage{courier}
\usepackage{listings}
\lstset{
    basicstyle=\footnotesize\ttfamily,
    breaklines=true,
    frame=single
}

% Címsorok
\usepackage{titlesec}
\titleformat{\section}{\bfseries\fontsize{14}{16}\selectfont}{\thesection}{1em}{}
\titleformat{\subsection}{\bfseries\fontsize{12}{14}\selectfont}{\thesubsection}{1em}{}

\title{Jegyzőkönyv – Sportautók Weboldal Kódjának Áttekintése}
\author{}
\date{}

\begin{document}
\maketitle
\tableofcontents
\newpage
\maketitle

\section{Bevezetés}
Ez a jegyzőkönyv a felhasználó által megadott HTML alapú weboldal áttekintését tartalmazza, amely egy sportautó-konfigurátor jellegű felületet mutat be. A dokumentum célja a weboldal szerkezetének, főbb funkcióinak, valamint a kódrészletek szerepének ismertetése.

\section{A projekt célja}
A projekt célja egy olyan vizuálisan látványos, interaktív weboldal létrehozása, ahol különböző sportautók adatai, képei, színei és konfigurációs lehetőségei böngészhetők.

\section{A főbb technológiák}
\begin{itemize}
    \item HTML5 – a weboldal szerkezete
    \item CSS – stilisztika és elrendezés
    \item Font Awesome – ikonok megjelenítése
    \item Swiper.js – képgaléria és slide-effektek
    \item JavaScript – funkcionalitások vezérlése
\end{itemize}

\section{Kódszerkezet áttekintése}

\subsection{Fejléc (Header)}
A weboldal egy mobilbarát, minimalista fejlécet alkalmaz.

\begin{verbatim}
<header class="header">
    <i class="fa-solid fa-bars"></i>
    <nav>
        <ul>
            <li><a href="#"><i class="fa-solid fa-magnifying-glass"></i></a></li>
            <li><a href="#">EN</a></li>
        </ul>
    </nav>
</header>
\end{verbatim}

\subsection{Fő tartalom (Hero Section)}
A főoldalon egy Lamborghini logó, címek és egy lejátszás gomb található.

\begin{verbatim}
<div class="content">
    <img src="kepek/Lamborghini-Logo.png" class="logo">
    <div class="title">
        <h1>Tervezd meg a saját</h1>
        <h1>Revueltódat</h1>
        <button class="icon-btn" onclick="showTrailer()">
            <i class="fa-solid fa-play"></i>
        </button>
    </div>
</div>
\end{verbatim}

\subsection{Autógaléria (Swiper)}
A weboldal központi eleme a Swiper.js segítségével létrehozott csúszka.

Példa slide:

\begin{verbatim}
<div class="swiper-slide">
    <div class="details-inner">
        <h1 class="sec-title">Engine</h1>
        <div class="details-container">
            <div class="sec">
                <div class="text-line">
                    <h3>HORSEPOWER</h3>
                    <p class="body-font">819 HP</p>
                </div>
            </div>
        </div>
    </div>
    <img src="kepek/reveulto.webp" alt="" />
</div>
\end{verbatim}

\subsection{Színválasztó szekció}
A felhasználó több szín közül választhat.

\begin{verbatim}
<div class="shape-line">
    <div class="shape c-2"></div>
    <span class="tin-font">giallo Auge</span>
</div>
\end{verbatim}

\subsection{Thumbnail slider}
A csúszkák alatt kis képes navigációs sáv található.

\begin{verbatim}
<div class="thumbs">
    <div class="swiper thumbsSwiper">
        <div class="swiper-wrapper">
            <div class="swiper-slide"><img src="kepek/reveulto.webp"></div>
        </div>
    </div>
</div>
\end{verbatim}

\section{Overlay és videólejátszás}
A weboldal tartalmaz egy videó overlay rendszert.

\begin{verbatim}
<div class="overlay">
    <i class="fa-solid fa-times" onclick="closeOverlay();"></i>
    <div id="car-video"></div>
</div>
\end{verbatim}

\section{JavaScript használata}
A script.js kezeli:
\begin{itemize}
    \item az overlay megnyitását és bezárását,
    \item a Swiper beállításait,
    \item funkcionális gombok működését.
\end{itemize}


\section{Összegzés}
A bemutatott projekt modern, felhasználóbarát felépítésű, moduláris kódszerkezettel. A Swiper integráció látványos galériát biztosít, a videó overlay pedig növeli az interaktivitást.

\end{document}

